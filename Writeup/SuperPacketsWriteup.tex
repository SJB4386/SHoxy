\documentclass[11pt]{article}
\usepackage{setspace}
\usepackage[letterpaper,top=1in,bottom=1in,left=1in,right=1in]{geometry}
\usepackage[parfill]{parskip}
\usepackage{listings}
\usepackage{arydshln}


\title{Super-Packets High-Availability Cluster Project}
\author{William Johnson, Spencer Berg,  Jonathan Rogers}
\date{7 March 2018}

\begin{document}
\maketitle

\thispagestyle{empty}

\begin{doublespace}



\end{doublespace}

\section{How the Proxy Handles Client Requests}
\subsection{Receiving a Request from the Client}
The SHOXY proxy server works by first initializing a thread that listens for connections and then initializing the threads that handle caching. Then the client must configure their browser settings to work through the SHOXY proxy. The client can then use their browser to reach websites by simply typing a URL into their address bar. The browser then sends the packets meant for the destination server to the SHOXY proxy. 

\subsection{Forwarding a Request to Intended Destination}
When the proxy receives a request from a client the proxy immediately creates a new thread which handles the logic of the request and goes back to listening for new client requests. The thread first checks to see if the request is HTTP GET which all requests from the client should be. Then the thread checks if the URL has been cached. If there is a cached version of the request then the SHOXY proxy responds back to the client with the cached version of the requested site. If no cached version of the request exists then the SHOXY proxy creates a new request from the forwarding data from the client's original request. This creates a copy of the client request but the header will contain the SHOXY proxy as the sender not the original client. 

\subsection{Receiving Server Response}
The destination server could return a myriad of responses. SHOXY proxy is built to handle HTTP codes 304, 400, and 501. 

\subsection{Sending Response Back to the Client}

\section{Caching}
\section{Code}

\begin{tiny}

\begin{lstinputlisting}[language=Java]{../src/SHoxy/Cache/CacheCleaner.java}
\end{lstinputlisting}
\begin{lstinputlisting}[language=Java]{../src/SHoxy/Cache/CachedItem.java}
\end{lstinputlisting}
\begin{lstinputlisting}[language=Java]{../src/SHoxy/Cache/CacheUpdater.java}
\end{lstinputlisting}
\begin{lstinputlisting}[language=Java]{../src/SHoxy/HTTP/HTTPData.java}
\end{lstinputlisting}
\begin{lstinputlisting}[language=Java]{../src/SHoxy/HTTP/HTTPEncoderDecoder.java}
\end{lstinputlisting}
\begin{lstinputlisting}[language=Java]{../src/SHoxy/Proxy/HTTPRequestHandler.java}
\end{lstinputlisting}
\begin{lstinputlisting}[language=Java]{../src/SHoxy/Proxy/ShoxyProxy.java}
\end{lstinputlisting}
\begin{lstinputlisting}[language=Java]{../src/SHoxy/Proxy/TCPListener.java}
\end{lstinputlisting}
\begin{lstinputlisting}[language=Java]{../src/SHoxy/Util/SHoxyUtils.java}
\end{lstinputlisting}

\end{tiny}

\end{document}
